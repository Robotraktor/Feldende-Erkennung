\documentclass[12pt,a4paper]{article}
\usepackage[utf8]{inputenc}
\usepackage[german]{babel}
\usepackage{graphicx}
\usepackage{amsmath}
\usepackage{amsfonts}
\usepackage{amssymb}
\usepackage[left=2cm,right=2cm,top=2cm,bottom=2cm]{geometry}

\author{Magdalena Klausner}
\usepackage[T1]{fontenc}
\newcommand{\changefont}[3]{
\fontfamily{#1}
\fontseries{#2}
\fontshape{#3}
\selectfont}
\changefont{cmss}{m}{n}

\begin{document}

\changefont{cmss}{m}{n}

\title{ Pflichtenheft}
\author{Auer Magdalena, Klausner Magdalena, Kostenzer Thomas }
\date{24.9.2014}
\centering\maketitle{5BHW-PPM}

\newpage
\tableofcontents
\newpage


\section{ Zielbestimmung}
\subsection{ Musskriterien:}

\begin{itemize}
\item Der Roboter kann geradeaus fahren 
\item Der Roboter kann das Ende des Feldes erkennen und dort stehen bleiben
\item Der Roboter kann Rechts- und Linkskurven fahren
\end{itemize}


\subsection{Wunschkriterien:}
\begin{itemize}
\item Der Roboter kann durch Abfahren des Feldes seine Fläche berechnen 
\item Der Roboter kann durch Referenzpunkte gerade fahren
\item Der Roboter kann Hindernisse erkennen und darauf reagieren  --\textgreater stehen bleiben oder ausweichen
\item Der Roboter sucht bei schwachem Akku seine Ladestation automatisch auf 
\end{itemize}


\section{ Produkteinsatz}

\subsection{Anwendungsbereiche:}

Anwendungsbereich ist ein Feld welches automatisch bearbeitet werden soll (Zurzeit wird nur ein kleiner Prototyp hergestellt)


\subsection{ Zielgruppen:}

Agrarwirtschaftliche Betriebe, die ihre Felder automatisch bearbeiten lassen möchten.


\subsection{ Betreibsbedingungen:}
\begin{itemize}
\item Betriebsdauer: wird später nachgetragen (hängt vom Akku ab)
\end{itemize}



\section{ Produkumgebung}
\subsection{Software:}
\begin{itemize}
\item Programmierumgebung: Arduino IDE (Version 1.0.3)
\item Programmiersprache: C++
\end{itemize}
	
\subsection{Hardware:}
\begin{itemize}
\item Mikrocontroller Arduino(Seeeduino)
\item 2 Servo-Motoren + Treiber
\item 3 Räder 
\item Sensor
\item Kabel
\item Batterie
\end{itemize}

\subsection{Orgware:}
\begin{itemize}
\item Blatt Papier
\item 2 verschiedene Farben (Schwarz und Weiß)
\end{itemize}



\section{ Produktfunktionen}

Der Robor soll als Erleichterung für Landwirte dienen. Sie müssen beispielsweise beim Pflügen den Traktor nicht mehr selbst bedienen, denn die Maschine kann die Arbeit automatisch verichten.
Wir arbeiten jedoch nur an einem Prototyp, der auf einem Blatt Papier (simuliertes Feld) eine markierte Fläche abfährt. 
Dabei soll er durch den Lichtsensor den Farbunterschied erkennen und entsprechend handeln.
Eventuell erfolgt eine Erweiterung, bei der beim Abfahren des Feldes die Fläche ausgemesssen wird, oder auch Hindernisse erkannt werden.
Eine weiterer Zusatz könnte sein, dass durch einen Referenzpunkt am Rand des Feldes der Roboter seine Spur halten kann.
Der Roboter bestitzt einen Akku(Laufzeit!!!), der entweder manuell oder über eine Ladestation aufgeladen wird.

\section{ Qualitätsbestimmung}

\begin{tabular}{l||c|c|c|c}

 &sehr wichtig & wichtig& weniger wichtig& unwichtig\\
 \hline
 \hline
 Hardware&&& X & \\
 \hline
 Programmkorrektheit& X &  & &\\
 \hline
 Zuverlässigkeit& X &  &  & \\
 \hline
 Benutzerfreundlichkeit& & &X &\\
 \hline
 Design&&&&X\\
 \hline
 Messgenauigkeit&&X&&\\
 
\end{tabular}







\end{document}
